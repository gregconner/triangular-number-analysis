
\documentclass{article}
\usepackage{amsmath}
\usepackage{amssymb}
\usepackage{geometry}
\geometry{margin=1in}

\title{Derivation of $(T_{T_m^2})^2$ Using Triangular Number Identity}
\author{Gregory Conner}
\date{\today}

\begin{document}

\maketitle

\section{Introduction}

This document derives mathematical expressions for $(T_{T_m^2})^2$ using the fundamental triangular number identity:

\begin{equation}
i^2 = T_{i-1} + T_i
\end{equation}

where $T_n = \frac{n(n+1)}{2}$ is the $n$-th triangular number.

\section{The Triangular Number Identity}

The key identity states that any perfect square can be expressed as the sum of two consecutive triangular numbers:

\begin{align}
i^2 &= T_{i-1} + T_i \\
&= \frac{(i-1)i}{2} + \frac{i(i+1)}{2} \\
&= \frac{i}{2}[(i-1) + (i+1)] \\
&= \frac{i}{2}[2i] \\
&= i^2
\end{align}

\section{Derivation of $(T_{T_m^2})^2$}

We want to find an expression for $(T_{T_m^2})^2$ using only terms of the form $T_l$.

\subsection{Step 1: Express $T_m$ and $T_m^2$}

For any natural number $m$:
\begin{align}
T_m &= \frac{m(m+1)}{2} \\
T_m^2 &= \left(\frac{m(m+1)}{2}\right)^2
\end{align}

\subsection{Step 2: Apply the identity to $T_m^2$}

Since $T_m^2$ is a perfect square, we can apply the identity $i^2 = T_{i-1} + T_i$:

Let $i = T_m$, then:
\begin{equation}
T_m^2 = T_{T_m-1} + T_{T_m}
\end{equation}

\subsection{Step 3: Calculate $T_{T_m^2}$}

Now we need to find $T_{T_m^2} = T_{T_{T_m-1} + T_{T_m}}$:

\begin{align}
T_{T_m^2} &= T_{T_{T_m-1} + T_{T_m}} \\
&= \frac{(T_{T_m-1} + T_{T_m})(T_{T_m-1} + T_{T_m} + 1)}{2}
\end{align}

\subsection{Step 4: Apply the identity to $(T_{T_m^2})^2$}

To find $(T_{T_m^2})^2$, we apply the identity again. Let $j = T_{T_m^2}$, then:

\begin{equation}
(T_{T_m^2})^2 = j^2 = T_{j-1} + T_j = T_{T_{T_m^2}-1} + T_{T_{T_m^2}}
\end{equation}

\section{Specific Cases}

\subsection{Case 1: $m = 1$}

\begin{align}
T_1 &= \frac{1 \cdot 2}{2} = 1 \\
T_1^2 &= 1^2 = T_0 + T_1 = 0 + 1 = 1 \\
T_{T_1^2} = T_1 &= 1 \\
(T_{T_1^2})^2 &= 1^2 = T_0 + T_1 = 0 + 1 = 1
\end{align}

Therefore: $(T_{T_1^2})^2 = T_0 + T_1$

\subsection{Case 2: $m = 2$}

\begin{align}
T_2 &= \frac{2 \cdot 3}{2} = 3 \\
T_2^2 &= 3^2 = T_2 + T_3 = 3 + 6 = 9 \\
T_{T_2^2} = T_9 &= \frac{9 \cdot 10}{2} = 45 \\
(T_{T_2^2})^2 &= 45^2 = T_{44} + T_{45}
\end{align}

Therefore: $(T_{T_2^2})^2 = T_{44} + T_{45}$

\subsection{Case 3: $m = 3$}

\begin{align}
T_3 &= \frac{3 \cdot 4}{2} = 6 \\
T_3^2 &= 6^2 = T_5 + T_6 = 15 + 21 = 36 \\
T_{T_3^2} = T_{36} &= \frac{36 \cdot 37}{2} = 666 \\
(T_{T_3^2})^2 &= 666^2 = T_{665} + T_{666}
\end{align}

Therefore: $(T_{T_3^2})^2 = T_{665} + T_{666}$

\section{General Pattern}

For any natural number $m$:

\begin{enumerate}
\item $T_m^2 = T_{T_m-1} + T_{T_m}$
\item $T_{T_m^2} = T_{T_{T_m-1} + T_{T_m}}$
\item $(T_{T_m^2})^2 = T_{T_{T_m^2}-1} + T_{T_{T_m^2}}$
\end{enumerate}

\section{Conclusion}

Using the triangular number identity $i^2 = T_{i-1} + T_i$, we can express $(T_{T_m^2})^2$ entirely in terms of triangular numbers:

\begin{equation}
(T_{T_m^2})^2 = T_{T_{T_m^2}-1} + T_{T_{T_m^2}}
\end{equation}

This demonstrates the power of the triangular number identity in reducing complex expressions involving squares to sums of triangular numbers.

\end{document}
