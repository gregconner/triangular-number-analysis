
\documentclass{article}
\usepackage{amsmath}
\usepackage{amssymb}
\usepackage{geometry}
\geometry{margin=1in}

\title{Complete Derivation of $(T_{T_m^2})^2$ Using Recursive Triangular Identity}
\author{Gregory Conner}
\date{\today}

\begin{document}

\maketitle

\section{Introduction}

This document derives mathematical expressions for $(T_{T_m^2})^2$ using the fundamental triangular number identity recursively:

\begin{equation}
i^2 = T_{i-1} + T_i
\end{equation}

where $T_n = \frac{n(n+1)}{2}$ is the $n$-th triangular number.

The key insight is that we must apply this identity recursively until no squares remain on the right side of the equation.

\section{The Triangular Number Identity}

The triangular number identity states that any perfect square can be expressed as the sum of two consecutive triangular numbers:

\begin{align}
i^2 &= T_{i-1} + T_i \\
&= \frac{(i-1)i}{2} + \frac{i(i+1)}{2} \\
&= \frac{i}{2}[(i-1) + (i+1)] \\
&= \frac{i}{2}[2i] \\
&= i^2
\end{align}

\section{Complete Recursive Derivation}

We want to find an expression for $(T_{T_m^2})^2$ using only terms of the form $T_l$, with no squares remaining.

\subsection{Step 1: Apply Identity to $T_m^2$}

For any natural number $m$:
\begin{align}
T_m &= \frac{m(m+1)}{2} \\
T_m^2 &= \left(\frac{m(m+1)}{2}\right)^2
\end{align}

Since $T_m^2$ is a perfect square, we apply the identity $i^2 = T_{i-1} + T_i$ where $i = T_m$:

\begin{equation}
T_m^2 = T_{T_m-1} + T_{T_m}
\end{equation}

\subsection{Step 2: Calculate $T_{T_m^2}$}

Now we calculate:
\begin{equation}
T_{T_m^2} = T_{T_{T_m-1} + T_{T_m}}
\end{equation}

This gives us the triangular number whose subscript is the sum of two consecutive triangular numbers.

\subsection{Step 3: Apply Identity to $(T_{T_m^2})^2$}

To find $(T_{T_m^2})^2$, we apply the identity again. Let $j = T_{T_m^2}$, then:

\begin{equation}
(T_{T_m^2})^2 = j^2 = T_{j-1} + T_j
\end{equation}

\subsection{Step 4: Eliminate Remaining Squares}

Now we must check if $j-1$ or $j$ are perfect squares and apply the identity recursively:

\begin{itemize}
\item If $j-1 = k^2$ for some $k$, then $T_{j-1} = T_{k-1} + T_k$
\item If $j = l^2$ for some $l$, then $T_j = T_{l-1} + T_l$
\end{itemize}

We continue this process until no squares remain in any subscript.

\section{Specific Cases}

\subsection{Case 1: $m = 1$}

\begin{align}
T_1 &= 1 \\
T_1^2 &= 1^2 = T_0 + T_1 = 0 + 1 = 1 \\
T_{T_1^2} = T_1 &= 1 \\
(T_{T_1^2})^2 &= 1^2 = T_0 + T_1 = 0 + 1 = 1
\end{align}

Since neither $0$ nor $1$ are perfect squares (except $1 = 1^2$, but $T_0 = 0$), we have:
\begin{equation}
(T_{T_1^2})^2 = T_0 + T_1
\end{equation}

\subsection{Case 2: $m = 2$}

\begin{align}
T_2 &= 3 \\
T_2^2 &= 3^2 = T_2 + T_3 = 3 + 6 = 9 \\
T_{T_2^2} = T_9 &= 45 \\
(T_{T_2^2})^2 &= 45^2 = T_{44} + T_{45}
\end{align}

Since neither $44$ nor $45$ are perfect squares, we have:
\begin{equation}
(T_{T_2^2})^2 = T_{44} + T_{45}
\end{equation}

\subsection{Case 3: $m = 3$}

\begin{align}
T_3 &= 6 \\
T_3^2 &= 6^2 = T_5 + T_6 = 15 + 21 = 36 \\
T_{T_3^2} = T_{36} &= 666 \\
(T_{T_3^2})^2 &= 666^2 = T_{665} + T_{666}
\end{align}

Since neither $665$ nor $666$ are perfect squares, we have:
\begin{equation}
(T_{T_3^2})^2 = T_{665} + T_{666}
\end{equation}

\section{General Pattern}

For any natural number $m$, the complete recursive application of the triangular number identity gives:

\begin{enumerate}
\item $T_m^2 = T_{T_m-1} + T_{T_m}$
\item $T_{T_m^2} = T_{T_{T_m-1} + T_{T_m}}$
\item $(T_{T_m^2})^2 = T_{T_{T_m^2}-1} + T_{T_{T_m^2}}$
\item If $T_{T_m^2}-1$ or $T_{T_m^2}$ are perfect squares, apply the identity recursively
\end{enumerate}

\section{Conclusion}

Using the triangular number identity $i^2 = T_{i-1} + T_i$ recursively, we can express $(T_{T_m^2})^2$ entirely in terms of triangular numbers with no squares remaining:

\begin{equation}
(T_{T_m^2})^2 = T_{T_{T_m^2}-1} + T_{T_{T_m^2}}
\end{equation}

This demonstrates the power of recursive application of the triangular number identity in reducing complex expressions involving multiple squares to sums of triangular numbers.

\end{document}
