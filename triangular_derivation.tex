
\documentclass{article}
\usepackage{amsmath}
\usepackage{amssymb}
\usepackage{geometry}
\geometry{margin=1in}

\title{Derivation of $(T_{T_m^2})^2$ Using Consecutive Triangular Sum Property}
\author{Gregory Conner}
\date{\today}

\begin{document}

\maketitle

\section{Introduction}

This document derives mathematical expressions for $(T_{T_m^2})^2$ using the fundamental property of triangular numbers:

\begin{equation}
T_n + T_{n+1} = (n+1)^2
\end{equation}

where $T_n = \frac{n(n+1)}{2}$ is the $n$-th triangular number.

\section{The Consecutive Triangular Sum Property}

The consecutive triangular sum property states that the sum of two consecutive triangular numbers equals the square of the larger subscript:

\begin{align}
T_n + T_{n+1} &= \frac{n(n+1)}{2} + \frac{(n+1)(n+2)}{2} \\
&= \frac{(n+1)}{2}[n + (n+2)] \\
&= \frac{(n+1)}{2}[2n+2] \\
&= \frac{(n+1)}{2} \cdot 2(n+1) \\
&= (n+1)^2
\end{align}

\section{Derivation of $(T_{T_m^2})^2$}

We want to find an expression for $(T_{T_m^2})^2$ using the consecutive sum property.

\subsection{Step 1: Express $T_m$ and $T_m^2$}

For any natural number $m$:
\begin{align}
T_m &= \frac{m(m+1)}{2} \\
T_m^2 &= \left(\frac{m(m+1)}{2}\right)^2 = \frac{m^2(m+1)^2}{4}
\end{align}

\subsection{Step 2: Calculate $T_{T_m^2}$}

\begin{align}
T_{T_m^2} &= \frac{T_m^2(T_m^2 + 1)}{2} \\
&= \frac{\frac{m^2(m+1)^2}{4}\left(\frac{m^2(m+1)^2}{4} + 1\right)}{2} \\
&= \frac{m^2(m+1)^2}{8}\left(\frac{m^2(m+1)^2}{4} + 1\right) \\
&= \frac{m^2(m+1)^2}{8} \cdot \frac{m^2(m+1)^2 + 4}{4} \\
&= \frac{m^2(m+1)^2[m^2(m+1)^2 + 4]}{32}
\end{align}

\subsection{Step 3: Apply Consecutive Sum Property}

If $T_{T_m^2}$ is a perfect square, say $T_{T_m^2} = k^2$ for some $k$, then we can apply the consecutive sum property:

\begin{equation}
T_{T_m^2} = T_{k-1} + T_k = k^2
\end{equation}

Therefore:
\begin{equation}
(T_{T_m^2})^2 = (k^2)^2 = k^4
\end{equation}

\subsection{Step 4: General Expression}

For the general case, we have:
\begin{equation}
(T_{T_m^2})^2 = \left(\frac{m^2(m+1)^2[m^2(m+1)^2 + 4]}{32}\right)^2
\end{equation}

\section{Specific Cases}

\subsection{Case 1: $m = 1$}

\begin{align}
T_1 &= \frac{1 \cdot 2}{2} = 1 \\
T_1^2 &= 1 \\
T_{T_1^2} = T_1 &= 1 = 1^2
\end{align}

Since $T_1 = 1^2$, we can apply the consecutive sum property:
\begin{align}
T_1 &= T_0 + T_1 = 0 + 1 = 1 \\
(T_{T_1^2})^2 &= (T_1)^2 = (1^2)^2 = 1^4 = 1
\end{align}

\subsection{Case 2: $m = 2$}

\begin{align}
T_2 &= \frac{2 \cdot 3}{2} = 3 \\
T_2^2 &= 9 \\
T_{T_2^2} = T_9 &= \frac{9 \cdot 10}{2} = 45
\end{align}

Since $T_9 = 45$ is not a perfect square, we cannot directly apply the consecutive sum property. The general formula gives:
\begin{equation}
(T_{T_2^2})^2 = (T_9)^2 = 45^2 = 2025
\end{equation}

\section{Conclusion}

The consecutive triangular sum property $T_n + T_{n+1} = (n+1)^2$ provides a powerful tool for deriving expressions involving triangular numbers. For $(T_{T_m^2})^2$:

\begin{itemize}
\item When $T_{T_m^2}$ is a perfect square, we can express $(T_{T_m^2})^2$ as $k^4$ where $k^2 = T_{T_m^2}$
\item In the general case, we have the explicit formula involving $m^2(m+1)^2$
\item The case $m = 1$ is special, allowing direct application of the consecutive sum property
\end{itemize}

This derivation demonstrates the deep connections between triangular numbers and perfect squares, revealing the elegant mathematical structure underlying these sequences.

\end{document}
